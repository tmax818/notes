% Created 2025-12-23 Tue 17:50
% Intended LaTeX compiler: pdflatex
\documentclass[11pt]{article}
\usepackage[utf8]{inputenc}
\usepackage[T1]{fontenc}
\usepackage{graphicx}
\usepackage{longtable}
\usepackage{wrapfig}
\usepackage{rotating}
\usepackage[normalem]{ulem}
\usepackage{amsmath}
\usepackage{amssymb}
\usepackage{capt-of}
\usepackage{hyperref}
\author{Tyler Maxwell}
\date{\today}
\title{notes index.org<notes>}
\hypersetup{
 pdfauthor={Tyler Maxwell},
 pdftitle={notes index.org<notes>},
 pdfkeywords={},
 pdfsubject={},
 pdfcreator={Emacs 29.3 (Org mode 9.6.15)}, 
 pdflang={English}}
\begin{document}

\maketitle
\tableofcontents




\section{introduction\hfill{}\textsc{notes}}
\label{sec:org25590f8}

In the interest of having a mimimum number of "buckets," This document will serve as a reference for notes. It seems the need for a separate directory for some topics rises immediately (i.e. \ref{sec:org9ffccd3}). 



\section{topics}
\label{sec:orge634d56}

\subsection{modern philosophy\hfill{}\textsc{book}}
\label{sec:orga776c74}
I like to (book:Scruton, Roger, 2012) 
\subsection{\href{content/worldOfChemistry/index.org}{world of chemistry}\hfill{}\textsc{book}}
\label{sec:org51b1d8d}
\subsection{\href{content/chemistry/index.org}{chemistry}\hfill{}\textsc{science:teaching}}
\label{sec:org1ae915c}
\subsubsection{\ref{sec:org0f2db13}}
\label{sec:orgd317116}
\subsection{education\hfill{}\textsc{education}}
\label{sec:org70277f3}
\subsection{\href{content/elisp/index.org}{elisp}\hfill{}\textsc{emacs}}
\label{sec:org3e59d06}
\subsubsection{\href{eintr}{intro to emacs}}
\label{sec:org5cbf4d7}
\subsection{haskell\hfill{}\textsc{coding}}
\label{sec:org9ffccd3}
\subsection{How to Take Smart Notes\hfill{}\textsc{book}}
\label{sec:orge30252e}
\begin{quote}
"One cannot think without writing."  
\end{quote}

\subsubsection{everything you need to know}
\label{sec:org8e34f64}
\subsubsection{everything you need to do}
\label{sec:org94e429d}
\subsubsection{everything you need to have}
\label{sec:org7541869}
\subsubsection{a few things to keep in mind}
\label{sec:orgda10da3}
\subsubsection{writing is the only thing that matters}
\label{sec:org5b73a5b}
\subsubsection{simplicity is paramount}
\label{sec:org2727c9d}
\subsubsection{nobody starts from scratch}
\label{sec:org9411a8a}
\subsubsection{let the work carry you forward}
\label{sec:org63bc132}
\subsubsection{separate and interlocking tasks}
\label{sec:org8abf60c}
\subsubsection{read for understanding}
\label{sec:orgc8e018a}
\subsubsection{take smart notes}
\label{sec:orgc9f6c09}
\subsubsection{develop ideas}
\label{sec:org96cb47c}
\subsubsection{share your insight}
\label{sec:org8233bef}
\subsubsection{make it a habit}
\label{sec:org11c3495}
\subsubsection{app}
\label{sec:orgc9f5e21}

\begin{quote}
Note sequences occupy the sweet spot between isolated facts/ideas and continuous text. They are what the Zettelkasten is all about. Therefore: Make sure a new note is written, whenever possible, in direct response to an existing note, continuing the dialogue.

If your program allows linking to individual blocks within a page, feel free to write whole note sequences on a single page with each block (or paragraph) as a single note. As long as you treat each block (or paragraph) as a stand-alone note, addressable from anywhere else, you keep the granularity of your notes while benefitting from a better overview over an existing thread.
\end{quote}

\subsection{org-mode\hfill{}\textsc{emacs}}
\label{sec:org1b192ab}
\subsubsection{\href{org}{the org manual}}
\label{sec:org3fd6ffa}
\begin{enumerate}
\item \href{org\#hyperlink}{hyperlinks}
\label{sec:org00aa877}
\end{enumerate}
\subsection{\href{content/physics/index.org}{physics}\hfill{}\textsc{science:teaching}}
\label{sec:org5616278}
\subsection{quotes\hfill{}\textsc{reading}}
\label{sec:orgef28636}
\subsubsection{Jesse Owens}
\label{sec:org0eb9078}
"The battles that count aren't the ones for gold medals. The struggle within yourself - the invisible, inevitable battles inside all of us - that's where it's at."
\subsubsection{Niklas Luhmann}
\label{sec:org6096308}
"One cannot think without writing."
\subsection{science\hfill{}\textsc{science}}
\label{sec:org8b4dbdb}
\subsection{teaching\hfill{}\textsc{teaching}}
\label{sec:org0f2db13}
\subsection{The Math of Life and Death\hfill{}\textsc{book}}
\label{sec:orgdb5a338}
\subsection{zettelkasten\hfill{}\textsc{notes}}
\label{sec:org2ad0d24}

\noindent
Scruton, Roger (2012). \emph{Modern Philosophy: A Survey}, Bloomsbury Reader.
\end{document}
